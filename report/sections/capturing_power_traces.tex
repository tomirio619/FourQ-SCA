% !TeX spellcheck = en_US
% !TeX root = ../Tom_Sandmann-master_thesis
\section{Capturing power traces}
To perform side channel analysis of the {\fourq} hardware design on the FPGA, we need to obtain power traces as {\fourq} is calculating the scalar multiplication.
A power trace is a collection of samples.
Each sample is a tuple of voltage and time values.
Time values are represented in seconds (s), while the amplitude values are represented in volts (v).
To obtain power traces, the FPGA is connected to the oscilloscope.
The SAKURA-G is designed with ultra-low noise in mind.
The board provides a couple of SMA connectors that can be used to monitor the power waveforms. 
The board also comes with an on-board amplifier, which can used to monitor the amplified waveform (for both the control and main FPGAs).
To control the acquisition of a power trace, we make use of a trigger.
The trigger tells the oscilloscope when it should start the acquisition of a waveform (i.e. when the value of the trigger is high) and when this acquisition should stop (i.e. when the oscilloscope's memory is full). 
The oscilloscope used to display and retrieve the captured power traces is the Teledyne LeCroy - WaveRunner 610Zi.
To connect with the oscilloscope and retrieve the acquired waveforms, an Ethernet cable (ENET) was employed.
Teledyne LeCroy oscilloscopes employ a standard Ethernet interface for utilizing the TCP/IP transport layer \cite{automation2017manual}.
Other methods for making the remote connection exist as well (such as USBTMC, GPIB and LSIB).
To interface with the oscilloscope, we make use of ActiveDSO, which is an ActiveX control.
ActiveDSO provides interface drivers and a client library to make the remote connection over ENET, GPIB or USBTMC interfaces.
It also supports many automation features besides remote control. One can read more about how Teledyne LeCroy oscilloscopes can be controlled by a variety of Windows applications and programming languages in the ActiveDSO's developer guide \cite{activedso2015guide}.
As the interface with the SAKURA-G board is written in Python, this is also the language of choice for communicating with the oscilloscope.
In Python, the control object used to communicate with the oscilloscope is instantiated as follows:
%
\begin{minted}[xleftmargin=\parindent, tabsize=4, obeytabs, breaklines, fontsize=\footnotesize]{python3}
command = "LeCroy.ActiveDSOCtrl.1"
_scope = win32com.client.Dispatch(command)
\end{minted}
%
Using the control object, we can write commands to the oscilloscope and read back the response. To make this possible, we connect to the oscilloscope:
%
\begin{minted}[xleftmargin=\parindent, tabsize=4, obeytabs, breaklines, fontsize=\footnotesize]{python3}
ip_address="192.168.0.1"
command = "IP:" + ip_address
_scope.MakeConnection(command)
\end{minted}
%
The IP address should match the IP address set in the settings of the oscilloscope.
After establishing a connection with the oscilloscope, we can use the control object to send commands to the device.
ActiveDSO supports two types of commands that can be sent to the oscilloscope using the instantiated control.
Both traditional IEEE 488.2 (GPIB) commands and the Windows\textsuperscript{\textregistered} Component Object Model (COM) commands can be used.
Examples of these commands are as follows:
%
\begin{minted}[xleftmargin=\parindent, tabsize=4, obeytabs, breaklines, fontsize=\footnotesize]{python3}
# 488.2 format
scope.WriteString("<command string>", <Boolean EOI>)
# Automation Control within the VBS command
scope.WriteString("app.Shutdown", True)
\end{minted}
%
If End of Identify (EOI) is set to 1 (\mintinline{text}{True}), the command terminates with EOI, and the device interprets the command right away. This is normally the desired behavior.
If EOI is set to 0 (\mintinline{text}{False}), a command may be sent in several parts with the device starting to interpret the command only when it receives the final part.
This final command should have set its EOI value set to (\mintinline{text}{True}).
If a command string contains characters like double quotes ("), the command string should be surrounded with triple quotes.
Otherwise, Python would interpret the first double quote as the end of the command string, which is unintended.

The oscilloscope offers a variety of interfaces for using devices to input analog or digital signals. 
A series of connectors arranged on the front of the instrument are used to input analog signals on channels 1-4. 
We use these analog inputs to connect our FPGA to the oscilloscope. 
Each of these channels interfaces power probes and completely integrates the probe with the channel. 
When connected, the probe type is recognized and some setup information, such as input coupling and attenuation, is performed automatically.
Besides these analog inputs, one can also make use of probes and the LBUS interface.
In our setup, we only use the analog inputs to capture the power traces of our FPGA.
To get the waveform from a corresponding channel, we have to call the appropriate ActiveDSO method.
The available methods for acquiring a waveform can be seen in \cite{activedso2015guide}.
%
%\begin{itemize}
%	\item \texttt{GetByteWaveform}. This method reads raw 8-bit waveform data from the instrument into a byte array. Visual Basic's (unsigned) byte type (values between 0 and 255) is used to store the signed data bytes (values between -128 and 127) that the scope emits. 
%	To make this work, the signed data is `shifted' by 128, such that it can fit in the unsigned data type. 
%	This should be remembered when the data is scaled.
%	The \texttt{GetByteWaveform} method should be used when unscaled 8-bit waveform data is required. 
%	Processed waveforms are usually 16-bit waveforms. 
%	To avoid losing precision, they should be transmitted in 16 bit form. 
%	This can be done by making use of the \texttt{GetIntegerWaveform} method.
%	
%	\item \texttt{GetIntegerWaveform}. This method reads raw 16-bit waveform data from the instrument into an integer array. 
%	This method should be used when unscaled 16-bit waveform data is required. 	
%	Processed waveforms are usually 16-bit waveforms. 
%	To avoid losing precision, they should be transmitted in 16-bit format.
%	Channel waveforms are usually 8-bit waveforms. 
%	They may be transferred using the \texttt{GetByteWaveform} method. 
%	This reduces the time and storage requirements.
%	
%	\item \texttt{GetNativeWaveform}. This methods reads a waveform from the instrument in its native binary form. 
%	You can specify whether the data should be transmitted as 16-bit words or as regular 8-bit bytes.
%	A parameter for this method is also used to specify which entity of the waveform should be transmitted.
%	Possible values fort his parameter are the descriptor (DESC), the user text (TEXT), the time descriptor (TIME), the data block (DAT1), a second block of data (DAT2) or all entities (ALL).
%	As mentioned in the previous described methods, processed waveforms are usually 16-bit waveforms and should be captured by setting the word data argument to \mintinline{text}{True}.
%	Channel waveforms ($C_1, \ldots, C_4$) should be transmitted by setting the word data argument to \mintinline{text}{False}.
%	It is also possible to restore a waveform captured using the \texttt{GetNativeWaveform}.
%	This is only possible if the waveform was captured using the (ALL) parameter.
%	If this is the case, the \texttt{SetNativeWaveformSetNativeWaveform} can be used to restore the waveform.
%	
%	\item \texttt{GetScaledWaveform}. This method reads a scaled waveform from the instrument. 
%	The result is a scaled waveform stored as an array of single-precision floating point values.
%	If the time value corresponding to each sample amplitude is required, the \texttt{GetScaledWaveformWithTimes} method should be used.
%	
%	\item \texttt{GetScaledWaveformWithTimes}. This method reads a scaled waveform from the instrument and stores the time and amplitude at each sample point.
%	The result is a scaled waveform stored as a two-dimensional array of single-precision floating point values. In the first column, the time values are stored. In the second column, the amplitude values are stored.
%	If the time value corresponding to each sample amplitude is not required, one can use the \texttt{GetScaledWaveform} method.
%	
%\end{itemize}
%%

%If 16-bit words are send from the oscilloscope, we have to take the order in which the bytes for a single word are stored into account.
%To determine which byte order the oscilloscope is currently using waveform data transmission, we can make use of the \mintinline{text}{COMM_ORDER} command. 
%This command can be used to either ask for the current communication order or to set the communication order.
%When setting the communication order using the \mintinline{text}{COMM_ORDER} command, we use the following command syntax: \mintinline{text}{COMM_ORDER <mode>}, with \mintinline{text}{<mode> := {HI, LO}}.
%If \mintinline{text}{HI} is used, waveform data is sent with the most significant byte (MSB) first (i.e. big endian). 
%If \mintinline{text}{LO} is used, waveform data is sent with the least significant byte (LSB) (i.e. little endian).
%As we can read in the Remote Control Manual, data should be sent with the LSB first for Intel-based computers.
%For Motorola-based computers, data should be sent with the MSB first, the default at power-up. Data written to the instrument's hard disk, USB, or floppy always remains in LSB first format (the default DOS format). You cannot use the \mintinline{shell}{COMM_ORDER} command in these cases, since it is only meant for data sent using GPIB and RS-232-C ports.
%
%Before acquiring a waveform using these methods, we can specify some additional information about the transfer of these waveform. 
%This is done by making use of the \texttt{SetupWaveformTransfer}.
%This method is used to configure various parameters that control the transfer of waveforms from the instrument to the PC.
%These arguments are as follows:
%%
%\begin{itemize}
%	\item \mintinline{text}{first_point}: An integer that describes the first point to transfer. A value of 0 indicates the first point.
%	
%	\item \mintinline{text}{sparsing}: An integer that describes the sparsing factor. This value indicate which values in the sample should be skipped.
%	A value of 0 indicates that all points should be transfered, while a value 2 means that every other point should be skipped.
%	
%	\item \mintinline{text}{segment_no}: An integer that specifies which segment number to transfer. A value of 0 indicates that all segments should be transferred.
%\end{itemize}
%%
%For the majority of these cases, the default values for these settings will be sufficient. This means that: all points are transferred, starting from the first point and transferring all segments.








