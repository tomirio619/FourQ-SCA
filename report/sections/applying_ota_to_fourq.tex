% !TeX spellcheck = en_US
% !TeX root = ../Tom_Sandmann-master_thesis
\section{\texorpdfstring{Applying online template attacks to \fourq}{Applying online template attacks to FourQ}}
In \Cref{algo: FourQ's scalar multiplication} in \Cref{chp: FourQ}, we can see the complete scalar multiplication of \fourq.
In order to retrieve the secret scalar, we need to obtain the $s_i$ and $d_i$ values that are applied in each iteration.
Recall that in the case of an OTA, the attacker can only capture one \emph{target trace} of the device under attack (i.e. the \emph{target device}).
In addition, we assume that the attacker knows the input point that belongs to this trace.
On the attacker's device, the attacker has full control over the implementation (which is the same implementation responsible for the captured \emph{target trace}).
This means that the attacker can change the scalar, but also the base point in the scalar multiplication.
We now describe how we apply an OTA to {\fourq}.
%
\begin{itemize}
	\item \textbf{Attacking the first digit-column}. 
	If we take a look at the actual scalar multiplication algorithm in \Cref{algo: FourQ's scalar multiplication}, we can see that $s_{64}$ determines the sign of the first element taken from the lookup table $T$.
	Which element is taken from this table is determined by the value $d_{64}$, which is unknown to us in the captured \emph{target trace}.
	After the initial assignment in \Cref{lst:fourq scalar mult:initial assignment}, the doubling operation of this initial value (i.e. loop at iteration $i = 63$) is done at \Cref{lst:fourq scalar mult:double oper}.
	This is the first doubling operation we are going to attack.
	If we take a look at the general scalar recoding algorithm employed by {\fourq} in \Cref{algo: Protected Recoding Algorithm for the GLV-SAC Representation} (in \Cref{chp: FourQ})%
	\footnote{In the case of \fourq, we have $l = 65$ and $m = 4$.}
	%
	, we can see that $b_{64}$ is always assigned a value of one, and that this value is used in the main loop as $s_{64}$.
	As we already know that the value of $s_{64} = 1$, we are left to guess the value of $d_{64}$, which is a 3-bit value.
	Our goal in the first `iteration' of our OTA is to try all of the possible $d_{64}$ values (i.e. our templates), such that we can obtain the corresponding power traces of the doubling operations.
	We then use these templates to find the one that matches best with the corresponding part of the \emph{target trace} (i.e. the first doubling operation).
	
	\item \textbf{Attacking the remaining digit-columns}. Once we have found the template (and its corresponding $d_{64}$ value) that matched the best with our \emph{target trace} at the first doubling operation, its time to attack the remaining digit-columns.
	At \Cref{lst:fourq scalar mult:add oper}, we see that both $s_i$ and $d_i$ determine the new value of the variable $Q$. 
	In the next iteration $i - 1$, this variable $Q$ is doubled again. To find out which values of $s_i$ and $d_i$ were used, we have to generate all of the possible templates for these values. This are at most 16 possible templates (3 bits for $d_i$ and 1 bit for $s_i$).
	Note that we need to compare each of the corresponding template traces at the $(i - 1)^\mathrm{th}$ doubling operation to attack the $d_i$ and $s_i$ values used in the addition operation in the $i^\mathrm{th}$ iteration.
	We are then iteratively constructing the matrix that corresponds to the recoded scalars of {\fourq}.
	Using these scalar values, we need to invert the scalar decomposition applied to the secret scalar to obtain the `original scalar' of the scalar multiplication that corresponds to the \emph{target trace}.  
	
	\item \textbf{Attacking the last digit-column}. We cannot use a doubling operation to attack the last digit-column and sign ($s_{0}$ and $d_0$) in the $0^\mathrm{th}$ iteration, as the main loop ends after this iteration. 
	There are however two other ways to attack the last digit-column and its corresponding sign. 
	One way is to use the generated templates for the addition operation involving $d_0$ and $s_0$, and use this operation instead to match the template traces with the target trace.
	Another way would be to brute force these last values, and see which output of the scalar multiplication gives the correct results.
\end{itemize}
%
As mentioned in \cite{batina2014online}, we can also use the addition operation in the $i^\mathrm{th}$ iteration together with the doubling operation in the $(i - 1)^\mathrm{th}$ iteration to attack digit-column $(s_i, d_i)$.
In this way, we can increase the chance of choosing the correct template.
This is however only possible when attacking the digit-columns not used in the first or last iteration (i.e digit-columns $(s_{63}, d_{63})$ up to $(s_{1}, d_{1})$).