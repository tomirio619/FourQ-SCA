% !TeX spellcheck = en_US
% !TeX root = ../Tom_Sandmann-master_thesis
\section{Efficient scalar multiplication}
Gallant--Lambert--Vanstone (GLV) method is a way of speeding up the computation of scalar multiplications on some elliptic curves defined over fields with a large prime characteristic.
Assume we have an elliptic curve of prime order $n$ with a point $P$.
By making use of the GLV method, we try to find a decomposition of the scalar multiplication $[k]P$ for $k \in \{1, \ldots, n\}$ into, for example, two scalar multiplications $[k]P = [u]P + [v]Q$.
This is a multi-exponentiation.
In general, a multi-exponentiation has the following form:
%
\begin{align*}
\sum_{i = 0}^{i < m} k_i P_i
\end{align*}
%
In the example we gave, the value of $m$ would be 2.
By rewriting this scalar multiplication, the new scalars $u, v$ only have half of the bitlength compared to the original bitlenght of the scalar $k$.
Such a scalar multiplication which involves two different points and two different scalars is computed by a \emph{double point multiplication} algorithm.
One obvious way to compute a double point multiplication is to perform two single-point multiplications. However, multiple algorithms exists that can compute $[u]P + [v]Q$ simultaneously and therefore more efficiently.
One of these algorithms is called the Straus-Shamir trick, which is an algorithm that can be used for simultaneous point multiplication.
The trick is in building one sequence of intermediate results that directly converge to the value of $[u]P + [v]Q$ in one execution of a Double-and-Add algorithm (which would be the traditional way of computing the scalar multiplication).
%https://books.google.nl/books?id=EQXnAwAAQBAJ&pg=PA174&lpg=PA174&dq=straus-shamir+trick+with&source=bl&ots=zaRlEXEtqD&sig=-0HGrNiuvnQ7qRNiu5Ac5_5Rgsg&hl=en&sa=X&ved=0ahUKEwjnvvfLmsHaAhXL-6QKHZ3pCYgQ6AEIXjAI#v=onepage&q=straus-shamir%20trick%20with&f=false
%http://cacr.uwaterloo.ca/hac/about/chap14.pdf
The algorithm can be seen in \Cref{algo: Straus-Shamir Trick} , where $\mathcal{S}$ is some set of integers containing 0 and 1 (as presented in \cite{moller2002improved}) and $\mathcal{E}$ an elliptic curve.
%
\begin{algorithm}
	\algorithmicrequire  $u = \sum_{i = 0}^{\ell - 1} u_i 2^i, v = \sum_{i = 0}^{\ell - 1} v_i 2^i, (u_{\ell - 1}, v_{\ell - 1}) \neq (0, 0), (u_i, v_i) \in \mathcal{S}^2, (P, Q) \in \mathcal{E}^2, P \neq \pm Q$.\\
	\algorithmicensure $R = [u] P + [v] Q$.
	%	
	\begin{algorithmic}[1]
		\State Precompute $W_{i, j} = [i]P + [j]Q, \forall(i, j) \in \mathcal{S} \setminus \{(0, 0)\}$
		\State Initialize $W_{u_{\ell -1}, v_{\ell - 1}}$
		\For {$i = \ell - 2$ \textbf{downto} $0$}
			\State $R = [2]R$
			\If{$(u_i, v_i) \neq (0, 0)$}
				\State $R = R + W_{u_i, v_i}$
			\EndIf
		\EndFor
	\end{algorithmic}
	%
	\captionof{algorithm}{Double Scalar Multiplication using the Straus-Shamir Trick \cite{rondepierre2013revisiting}.}
	\label{algo: Straus-Shamir Trick}
\end{algorithm}
%
The Straus-Shamir trick is actually a special case of Straus' algorithm, in which the window size is set to 1.
This trick reduces the number of doublings by half.
To further increase the performance, we can make use of signed digit representations and windowing.
By making use of these methods, we can increase the number of null digits of the scalar, which increases the performance by reducing the number of additions.
These methods require the scalar to be transformed to another representation, which is called \emph{scalar recoding}.
We discuss other methods of performing efficient scalar multiplications in the upcoming subsections.

\subsection{Comb method} \label{subsec: Comb method}
The comb method is a way to perform scalar multiplication.
The method assumes that the scalar $k$ is represented by a matrix of $w$ rows and $d$ columns.
During the scalar multiplication, $k$ is being processed columwise. 
Initially, the binary representation of $k$ is first padded on the left with $dw - t$ zeros, such that its length is $dw$.
$k$ is now split up into $w$-bit strings $K$ with each of these string having a length of $d$, such that:
%
\begin{equation*}
k = K^{w-1} \Vert \dots \Vert  K^{1} \Vert K^0
\end{equation*}
%
with $\Vert$ denoting the bitwise concatenation operator.
The bit strings $K^j$ are now written as rows of an \emph{exponent array} \cite{hankerson2006guide}:
%
\begin{equation*}
%
\begin{bmatrix}
		K^0 \\
		\vdots \\
		K^{w'} \\
		\vdots \\
		K^{w - 1}
\end{bmatrix}
%
=
%
\begin{bmatrix}
k_{d-1}^0 & \dots & k_0^0 \\
\vdots & & \vdots \\
K_{d-1}^{w'} & \dots & K_0^{w'} \\
\vdots & & \vdots \\
K_{d-1}^{w-1} & \dots & K_0^{w - 1}
\end{bmatrix}
%
=
%
\begin{bmatrix}
k_{d-1} & \dots & k_0 \\
\vdots & & \vdots \\
K_{(w'+1)d-1} & \dots & K_{w'd} \\
\vdots & & \vdots \\
K_{wd-1} & \dots & K_{(w-1)d}
\end{bmatrix}
\end{equation*}
%
The columns of this exponent array will then be processed one at a time from left to right during the scalar multiplication.
To speed up the computation of the scalar multiplication, the following points are precomputed \cite{hankerson2006guide}:
%
\begin{align*}
\left[ a_{w - 1}, \ldots, a_2, a_1, a_0 \right]P = a_{w-1} 2^{(w-1)d} P + \ldots + a_2 2^{2d} P + a_1 2^ d P + a_0 P
\end{align*}
%
This precomputation is done for all possible bit strings $(a_{w - 1}, \ldots, a_1, a_0)$.
The algorithm itself can be seen in \Cref{algo: Fixed-base comb method for point multiplication}.
%
\begin{algorithm}
	\algorithmicrequire Scalar $k = (k_{t-1}, \ldots, k_1, k_0)_2$, window width $w$, bit string size $d = \lceil t / w\rceil$ and $P \in E(\mathbb{F}_p)$.
	\algorithmicensure $R = [k]P$.
	%	
	\begin{algorithmic}[1]
		\Statex \textbf{Precomputation}
		\State Compute $[ a_{w - 1}, \ldots, a_2, a_1, a_0 ]P$ for all bit strings  $(a_{w - 1}, \ldots, a_1, a_0)$ with length $w$
		\State Write $k = K^{w-1} \Vert \dots \Vert  K^{1} \Vert K^0$ and add padding on the left if necessary. Each $K^j$ is a bit string of length $d$. With $K_i^j$ we denote the $i$-th bit of the bit string $K^j$.
			\State $R \gets \infty$
		\For {$i = d - 1$ \textbf{downto} $0$}
			\State $R \gets 2R$
			\State $R \gets R + \left[ K_i^{w-1}, \ldots, K_i^1, K_i^0 \right] P$
		\EndFor
		\State \textbf{Return} $Q$
	\end{algorithmic}
	%	
	\captionof{algorithm}{Fixed-base comb method for point multiplication \cite{hankerson2006guide}.}
	\label{algo: Fixed-base comb method for point multiplication}
\end{algorithm}
%

% NAF: https://crypto.stackexchange.com/a/25229
\subsection{Interleaving}
Another way of performing scalar multiplication is by making use of interleaving.
As we have seen with the Straus-Shamir trick, the precomputed values were obtained by making use of two points. 
When each precomputed value only involves one point, the associated method of multiple point multiplication is also known as \emph{interleaving}.
An example of such a method is an interleaving algorithm which makes use of the \textbf{non-adjacent form (NAF)} of a number.
The NAF of a number is a unique signed-digit representation, which means that non-zero values cannot be adjacent.
The integer $7$ in NAF would be represented as follows: $(1~0~0~\bar{1})$, where $\bar{1} = - 1$.
There exist different algorithms to convert a number to NAF, one of which is introduced in \cite{hankerson2006guide}.
If we want to transform an integer $k$ to NAF, this algorithm will repeatedly divide $k$ by 2. 
Remainders of $0$ or $\pm 1$ are allowed, but if $k$ is odd, then the remainder $r \in \{-1, 1\}$ is chosen in such a way that the quotient $(k - r) / 2$ is even.
This makes sure that the next NAF digit is 0.
The corresponding algorithm can be seen in \Cref{algo: compute NAF}.
% Two algorithms side by side, see https://tex.stackexchange.com/questions/418185/putting-two-algorithms-side-by-side-with-reasonable-margin
\begin{figure}
%
	\begin{minipage}[t]{7.0cm}%
		\vspace{2pt}
		\begin{algorithm}[H]
			\algorithmicrequire A positive integer $k$.\\
			\algorithmicensure $\text{NAF}(k)$.
			%	
			\begin{algorithmic}
				\State $i \gets 0$
				\While{$k \ge 1$}
				\If{$k$ is odd}
				\State $k_i \gets 2 - (k \pmod{4})$ \label{lst: Interleaving with NAFs: line change for w-width NAF}
				\State $k \gets k - k_i$
				\Else
				\State $k_i \gets 0$
				\EndIf
				\State $k \gets k / 2$
				\State $i \gets i + 1$
				\EndWhile
				\State \textbf{Return} $(k_{i - 1}, k_{i - 2}, \ldots, k_1, k_0)$
			\end{algorithmic}
			%
			\captionof{algorithm}{Computing the NAF of a positive integer \cite{hankerson2006guide}.}
			\label{algo: compute NAF}
		\end{algorithm}
		% @see Guide to Elliptic Curve Cryptography
	\end{minipage}%
	%
	\hfill%
	%
	\begin{minipage}[t]{7.0cm}%
		\vspace{2pt}
		\begin{algorithm}[H]
			\algorithmicrequire A positive integer $k$.\\
			\algorithmicensure $\text{NAF}_w(k)$.
			%	
			\begin{algorithmic}
				\State $i \gets 0$
				\While{$k \ge 1$}
				\If{$k$ is odd}
				\State $k_i \gets k \operatorname{mods}{2^w}$
				\State $k \gets k - k_i$
				\Else
				\State $k_i \gets 0$
				\EndIf
				\State $k \gets k / 2$
				\State $i \gets i + 1$
				\EndWhile
				\State \textbf{Return} $(k_{i - 1}, k_{i - 2}, \ldots, k_1, k_0)$
			\end{algorithmic}
			%
			\captionof{algorithm}{Computing the width-$w$ NAF of a positive integer \cite{hankerson2006guide}.}
			\label{algo: compute NAF-w}
		\end{algorithm}
		% @see Guide to Elliptic Curve Cryptography
	\end{minipage}%
	%
\end{figure}
An interleaving method for computing $\sum_{j = 1}^{v} k^j P_j$ can be seen in \Cref{algo: interleaving with NAFs}.
%
\begin{algorithm}
	\algorithmicrequire  $v$, a set of integers $k^j$, widths $w_j$ and points $P_j$, with $1 \le j \le v$.\\
	\algorithmicensure $\sum_{j = 1}^{v} k^j P_j$.
	%
	\begin{algorithmic}[1]
		\State Compute $iP_j$ for $i \in \{1, 3 , \ldots, w^{w_j - 1} - 1\}$ for $1 \le j \le v$
		\State Compute NAF$_{w_j} (k^j) = \sum_{i = 0}^{l_j - 1} k_i^j 2^i$ for $1 \le j \le v$
		\State $l \gets \text{max} \{l_j : 1 \le j \le v \}$ \Comment{$l_j$ denotes the length of NAF$_{w_j}(k^j)$}
		\State $k_i^j = 0$ for $l_j \le i < l, 1 \le j \le v$
		\State $R \gets \infty$
		\For{$i = l-1$ \textbf{downto} $0$}
			\State $R \gets 2R$
			\For{$j = 1$ \textbf{to} $v$}
				\If{$k_i^j \neq 0$}
					\If{$k_i^j > 0$}
						\State $R \gets R + k_i^j P_j$
					\Else
						\State $R \gets R - k_i^j P_j$
					\EndIf
				\EndIf	
			\EndFor
		\EndFor
		\State \textbf{Return} $R$
	\end{algorithmic}
	%
	\captionof{algorithm}{Interleaving with NAFs \cite{hankerson2006guide}.}
	\label{algo: interleaving with NAFs}
\end{algorithm}
%
The algorithm makes use of a window method which processes only a particular number of digits at a time.

If $w \ge 2$ is a positive integer, then a \emph{width}-$w$ \emph{NAF} of a positive integer $k$ is an expression $k = \sum_{i = 0}^{l - 1} k_i 2^i$ in which each nonzero coefficient $k_i$ is odd. 
In addition, $\abs{k_i} < 2^{w - 1}$ with $k_{l - 1} \neq 0$. 
At most one of any $w$ consecutive digits is nonzero. The length of the width-\emph{w} NAF is $l$. 
Computation of a $w$-width NAF requires a small change of \Cref{algo: compute NAF} at \Cref{lst: Interleaving with NAFs: line change for w-width NAF}. This can be seen in \Cref{algo: compute NAF-w}.
Instead of calculating $k_i \gets 2 - (k \pmod{4})$, the computation becomes $k_i \gets k \operatorname{mods}{2^w}$, where $ k \operatorname{mods}{2^w}$ denotes the integer $u$ which satisfies $u \equiv k \pmod{2^w}$ with $-2^{w-1} \le u < 2^{w - 1}$ \cite{hankerson2006guide}.

We now discuss the interleaving algorithm shown in \Cref{algo: interleaving with NAFs}.
We explain how this algorithm works in the case of double exponentiation (i.e. $v = 2$).
If we want to calculate $R = [k_0^j]P + [k_1^j] Q$, we first precompute $\{P, 3P, 5P, \ldots, [(2w - 1) / 2]P \}$ and $\{ Q, 3Q, 5Q, \ldots, [(2w - 1)] P \}$ for a choice of window size $w$. 
We then convert both $k_0^j$ and $k_1^j$ to width-$w$ NAF format. 
After initializing a point $R$ to the point-at-infinity, we scan the NAFs for $k_0^j$ and $k_1^j$ from left to right. 
While we process each bit, we double the value of $R$. 
As we are scanning the NAFs, we choose the subsections of the corresponding NAF to get the largest multiple of $P$ or $Q$.
This value is present in the tables we precomputed earlier.
We add this precomputed multiple of $P$ or $Q$ to $R$.
As the value of $w$ becomes bigger, more time is required for precomputation.
This also increases the amount of memory needed to store these precomputed tables. 
However, in the end this leads to less additions in the main double-and-add loop which increases the performance of the algorithm.

Multi-exponentiation as shown in the \emph{Interleaving with NAFs} algorithm can also be done by making use of a morphism (also called a homomorphism).
A morphism is a structure-preserving map from an mathematical structure to itself.
An endomorphism is a morphism from a mathematical object to itself.
An example of an endomorphism is the Frobenius endomorphism, which is a special endomorphism of commutative rings with prime characteristic $p$.
The Frobenius morphism maps every elements to its $p$-th power.
If we have commutative ring $R$ with prime characteristic $p$, the Frobenius endomorphism is defined as:
%
\begin{align*}
F(p) = r^p
\end{align*}
%
for all $r \in R$.
We can make use of homomorphisms in variable point multiplication.
Assume we want to calculate $[k]p = [u]P + [v]Q$, where $Q = \psi(P)$ and $\varphi$ being a homomorphism.
Again we have the precomputed table $\{P, 3P, 5P, \ldots, [(2w - 1) / 2]P \}$.
We can now easily compute the other table by applying $\psi$ to each of the elements in the table we already precomputed.
This speeds up the precomputation phase.
This approach is also adopted by {\fourq}, and is described in more detail in \Cref{chp: FourQ}.
