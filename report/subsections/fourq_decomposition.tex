% !TeX spellcheck = en_US
% !TeX root = ../Tom_Sandmann-master_thesis
\subsection{Scalar decomposition} \label{subsec: Scalar decomposition}
In this subsection, we describe how {\fourq} decomposes the scalar $m \in \mathbb{Z}$ into the corresponding 4-dimensional multiscalar $(a_1, a_2, a_3, a_4) \in \mathbb{Z}$ such that $m \equiv a_1 + a_2 \lambda_\phi + a_3 \lambda_\psi + a_4 \lambda_\phi \lambda_\psi \pmod{N}$ with $0 \le a_i < 2^{64} - 1$ for $i \in \{1,2,3,4\}$.
In this decomposition, $\lambda_\psi$ and $\lambda_\phi$ are the eigenvalues of the two endomorphisms $\psi$ and $\phi$ on $\mathcal{E}(\mathbb{F}_{p^2}) \left[ N \right]$, which are defined as follows \cite{costello2015fourq}:
%
\begin{align*}
\lambda_\psi &:= 4 \cdot \frac{p + 1}{r} \pmod{N} \\
\lambda_\phi & := 4 \cdot \frac{(p - 1) r^3}{(p + 1)^2 V} \pmod{N}
\end{align*}
%
with $r$ and $V$ being integers that satisfy the formulas presented in \cite[§3.3]{costello2015fourq}. 
We fix both integers by taking the following solution to these equations:
%
\begin{equation*}
V := 49293975489306344711751403123270296814, ~~~~ r := 15437785290780909242
\end{equation*}
%
To be able to decompose the scalar, we need to introduce of couple of definitions and concepts.
First, we define the \textit{lattice of zero decompositions} as follows \cite{costello2015fourq}:
%
\begin{equation*}
\mathcal{L} := \langle (z_1, z_2, z_3, z_4) \in  \mathbb{Z} \mid  z_1 + z_2 \lambda_\phi + z_3 \lambda_\psi + z_4 \lambda_\phi \lambda_\psi \equiv 0 \pmod{N} \rangle
\end{equation*}
%
Now the set of decompositions for $m \in \mathbb{Z} / N \mathbb{Z} $ (i.e. the ring of integers modulo $N$) is the lattice coset $(m, 0, 0, 0) + \mathcal{L}$.
Given a basis $\bm{B} = (\bm{b}_1, \bm{b}_2, \bm{b}_3, \bm{b}_4)$ of $\mathcal{L}$, we can now compute $(\alpha_1, \alpha_2, \alpha_3, \alpha_4) \in \mathbb{Q}^4$ as the unique solution to  $(m, 0, 0, 0) = \sum_{i=1}^{4} \alpha_i \bm{b}_i$ for any $m \in \mathbb{Z}$. This is done by making use of the \emph{Babai rounding} technique \cite{babai1986lovasz}.
The rounding technique is an alternative to the nearest plane method, which is a method used to solve the closest vector problem (CVP) \cite{galbraith2012mathematics}.
In general, the \emph{Babai rounding} method is not guaranteed to solve the closest vector problem, but in this case it will find the solution (see \cite[§4.1]{costello2015fourq}). 
Application of this method involves the computation of multiple roundings.
Therefore we need to account for possible rounding errors.
This is described later on.
The multiscalar now becomes as follows:
%
\begin{equation*}
(a_1, a_2, a_3, a_4) = (m, 0, 0,0 ) - \sum_{i=1}^{4} \lfloor \alpha_i \rceil \cdot \bm{b}_i
\end{equation*}
%
where $\lfloor \alpha_i \rceil$ indicates that $\alpha_i$ gets rounded to the nearest integer.
It follows that $(a_1, a_2, a_3, a_4) - (m, 0, 0, 0) \in \mathcal{L}$.
A Babai optimal basis $\bm{B} = (\bm{b}_1, \bm{b}_2, \bm{b}_3, \bm{b}_4)$ for the zero decomposition lattice is as follows \cite{costello2015fourq}:
%
\begin{align*}
224 \cdot \bm{b}_1 &:= \left( 16(-60\alpha + 13r - 10), 4(-10\alpha - 3r + 12), 4(-15 \alpha + 5r - 13), -13 \alpha - 6r + 3 \right) \\
8 \cdot \bm{b}_2 &:= (32(5\alpha - r), -8, 8, 2\alpha + r) \\
224 \cdot \bm{b}_3 &:= (16(80\alpha - 15r + 18), 4(18\alpha - 3r - 16), 4(-15 \alpha - 9r + 15), 15 \alpha + 8r + 3) \\
448 \cdot \bm{b}_4 &:= (16(-360 \alpha + 77r + 42),  4(42 \alpha + 17r + 72), 4(85\alpha - 21r - 77), (-77\alpha - 36r - 17))
\end{align*}
%
where $\alpha = V/r \in \mathbb{Z}$.
We have to note that in \cite{costello2015fourq}, a typo was made where the value of $\bm{b_3}[4]$ was defined as $15 \alpha + 8r + 3 \alpha$, which is incorrect.
With the optimal basis $\bm{B}$, the scalar decomposition using the Babai rounding technique becomes as follows \cite{costello2015fourq}:
%
\begin{align*}
(a_1, a_2, a_3, a_4) &= (m, 0, 0,0 ) - \sum_{i=1}^{4} \lfloor \alpha_i \rceil \cdot \bm{b}_i \\
6272r^3 \cdot \hat{\alpha}_1 &= 540V^3 + 10r(27r - 4)V^2 + 6r^2(9r^2 - 2r + 18)V + r^3(27r + 4)(r^2 - 2) \\
25088r^3 \cdot \hat{\alpha}_2 &= 1020V^3 + 10r(47r - 8)V^2 + 2r^2(51r^2 + 26r + 102)V + r^3(47r + 8)(r^2 - 2) \\
25088r^3 \cdot \hat{\alpha}_3 &= 220V^3 + 10r(11r + 16)V^2 + 2r^2(11r^2 - 46r + 22)V + r^3(11r - 16)(r^2 - 2) \\
1792r3 \cdot \hat{\alpha}_4 &= 60V^3 + 30r^2V^2 + 2r^2(3r^2 + 2r + 6)V + 3r^4(r^2 - 2) 
\end{align*}
%
where $\alpha_i = \hat{\alpha}_i \cdot m/N$.
We end up with the following decomposition for $m$:
%
\begin{align*}
m \equiv a_1 + a_2 \lambda_\phi + a_3 \lambda_\psi + a_4 \lambda_\phi \lambda_\psi \pmod{N}
\end{align*}
%
with $\abs{a_i} < 2^{62}$ for $i \in \{1,\ldots,4\}$.
As mentioned previously, the method of \emph{Babai rounding} used to find the multiscalar makes use of four roundings $\lfloor \frac{\hat{\alpha}_i}{N} \cdot m \rceil$, where $m$ is the input scalar and $N$ the fixed curve constant.
The four $\hat{\alpha}_i$ and their denominators $N$ are fixed curve constants.
An efficient rounding technique exist (as mentioned in \cite{costello2015fourq}), which works by choosing a power of 2 that is greater than the denominator $N$.
If we call this value $\mu$, we can use it to calculate the fixed curve constants $\ell_i = \lfloor \frac{\hat{\alpha}_i}{N} \cdot \mu \rceil$.
If we now want to calculate $\lfloor \frac{\hat{\alpha}_i}{N} \cdot m \rceil$, we can do it as follows:
%
\begin{equation*}
\left\lfloor \frac{\ell_i \cdot m}{\mu} \right\rceil = \left\lfloor \left\lfloor \frac{\hat{\alpha}_i \cdot \mu}{N} \right\rceil \cdot \frac{m}{\mu} \right\rceil
\end{equation*}
%
The division by $\mu$ at runtime can now easily be done by a simple bitshift.
However, this rounding method can sometimes yield results that are off by 1 \cite[§4.2 Lemma 1]{costello2015fourq}:
%
\begin{equation*}
\left\lfloor \frac{\ell_i \cdot m}{\mu} \right\rceil - \left\lfloor \left\lfloor \frac{\hat{\alpha}_i \cdot \mu}{N} \right\rceil \cdot \frac{m}{\mu} \right\rceil \in \{0, 1\}
\end{equation*}
%
In \cite[§4.2 Lemma 1]{costello2015fourq}, the value above is proven to be greater than $-1/2 - m/(2\mu)$ and less than $3/2 + m/2(\mu)$.
Larger values of $\mu$ decrease the chance of rounding errors, but in {\fourqs} design it was decided to always account for these rounding errors \cite{costello2015fourq}.
This is done by allowing:
%
\begin{equation*}
(a_1, a_2, a_3, a_4) = \sum_{i=1}^{4}(\alpha_i - \hat{\alpha}_i) \bm{b}_i = \sum_{i=1}^{4}(\alpha_i - (\hat{\alpha}_i - \epsilon_i)) \bm{b}_i
\end{equation*}
%
for all the possible combinations (sixteen in total) of $\epsilon_i \in \{0, 1\}$ and $i \in \{1, 2, 3, 4\}$.
This means that all integers less than $\mu$ will decompose to a multiscalar in $\mathbb{Z}^4$.
Each of the coordinates in this multiscalar lie inside the parallelepiped $\mathcal{P}_\epsilon (\bm{B}) := \{ \bm{B}x \mid x \in [-1/2, 3/2)^4\}$.
In addition, $\mu$ becomes the fixed value $2^{256}$.
Another thing that remains to be addressed is the sign of the multiscalar.
Many points in $\mathcal{P}_\epsilon (\bm{B}) \cap \mathbb{Z}^4 $ are far greater than $2^{62}$ in absolute value. 
In addition, the majority of these points will have coordinates that are both positive and negative.
Dealing with signed multiscalars can require an additional iteration in the main loop for the scalar multiplication, which is unwanted.
Therefore, an offset vector in $\mathcal{L}$ is used to find a translate of $\mathcal{P}_\epsilon (\bm{B})$  which only contains positive coordinates.
This is done by finding two vector $\bm{c}$ and $\bm{c}'$ in $\mathcal{L}$ that achieve this property.
This leads to the following multiscalar corresponding to the decomposition of $m \in [0, 2^{256})$ \cite{costello2015fourq}:
%
\begin{align*}
a_1 &= m - \tilde{a_1}    \cdot \bm{b_1}[1] - \tilde{a_2} \cdot \bm{b_2}[1] - \tilde{a_3} \cdot \bm{b_3}[1] - \tilde{a_4} \cdot \bm{b_4}[1] \\
a_2 &=~~~~~ - \tilde{a_1} \cdot \bm{b_1}[2] - \tilde{a_2} \cdot \bm{b_2}[2] - \tilde{a_3} \cdot \bm{b_3}[2] - \tilde{a_4} \cdot \bm{b_4}[2] \\
a_3 &=~~~~~ - \tilde{a_1} \cdot \bm{b_1}[3] - \tilde{a_2} \cdot \bm{b_2}[3] - \tilde{a_3} \cdot \bm{b_3}[3] - \tilde{a_4} \cdot \bm{b_4}[3] \\
a_4 &=~~~~~ - \tilde{a_1} \cdot \bm{b_1}[4] - \tilde{a_2} \cdot \bm{b_2}[4] - \tilde{a_3} \cdot \bm{b_3}[4] - \tilde{a_4} \cdot \bm{b_4}[4]
\end{align*}
%
where $\bm{B}$ is the basis introduced earlier, $\ell_i := \lfloor \hat{\alpha}_i \cdot \mu \ N \rceil$ are the four curve constants (with $\hat{\alpha}_i$ as defined earlier).
$\bm{c} = 5\bm{b}_2 - 3\bm{b}_3 + 2\bm{b}_4$ and $\bm{c'} = 5\bm{b}_2 - 3\bm{b}_3 + 3\bm{b}_4$ are the translation vectors used to make the coordinates positive, and $\tilde{\alpha}_i = \lfloor \ell_i m / \mu \rfloor$. In \cite[§4.3, Proposition 5]{costello2015fourq} and its corresponding proof, it is shown that both of the multiscalars $(a_1, a_2, a_3, a_4) + \bm{c}$ and $(a_1, a_2, a_3, a_4) + \bm{c'}$ are valid decompositions of $m$ (with all of the coordinates being positive and less than $2^{64}$). 
In addition, it is shown that exactly \emph{one} of these decompositions of $m$ has a first coordinate that is odd, which is required as it makes scalar multiplication and recoding easier.