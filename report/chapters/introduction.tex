% !TeX spellcheck = en_US
% !TeX root = ../Tom_Sandmann-master_thesis
\chapter{Introduction}
\lettrine[lhang = 0.4, findent=-30pt, lines=4]{\textbf{
		\initfamily \fontsize{20mm}{20mm} \selectfont I
		\normalfont}}{n}
public key cryptography, the cryptographic system has a pair of keys: the public and private key.
The public key can be distributed freely, while the private key must remain known only to the owner.
If this is the case, authentication, encryption and non-repudiation can be achieved.
We introduce to three major families of public-key algorithms based on their underlying computational problem \cite{paar2009understanding}:
%
\begin{itemize}
	\item \textbf{Integer-Factorization Schemes:} schemes based on the fact that large integers are hard to factor. An example of a scheme that falls into this category is RSA.
	
	\item  \textbf{Discrete Logarithm Schemes:} schemes based on the discrete logarithm problem in groups. An example of a scheme that falls into this category is the Diffie-Hellman key exchange.
	 
	\item \textbf{Elliptic Curve Schemes}: A generalization of the discrete logarithm algorithm are elliptic curve public-key schemes. An example of a scheme that falls into this category is the Elliptic Curve Diffie-Hellman key exchange (ECDH).
\end{itemize}
\parshape=0
%
An advantage of Elliptic Curve Cryptography (ECC) is the fact that it can offer the same level of security while using much smaller parameters than non-ECC cryptography.
This leads to a significant increase in performance and makes these algorithms more suitable for (embedded) systems where amount of memory is limited and where energy consumption should be minimal.

A Field-Programmable Gate Array (FPGA) is an integrated circuit that is designed to be configurable after it has been manufactured.
FPGAs have become an attractive option for deploying hardware applications in comparison to the well-established Application-Specific Integrated Circuits (ASICs).
Besides their great flexibility, FPGAs also reduce the development costs and allow for faster prototyping.
For these reasons, FPGAs have become targets for many ECC implementations \cite{guneysu2008ultra, sasdrich2014efficient, sasdrich2015implementing}.

In \cite{costello2015fourq}, a new elliptic curve with the name {\fourq} is proposed. 
This curve provides approximately 128 bits of security.
By combining a four-dimensional decomposition wit the fastest (explicit) twisted Edwards curves formulas available and in combination with the efficient Mersenne prime $p = 2^{127} - 1$, it supports highly-efficient scalar multiplications.
For generic scalar multiplications, {\fourq} performs four to five times faster than the original NIST P-256 curve \cite{costello2015fourq, gueron2015fast}, and is also faster than curves that were considered as NIST alternatives after its introduction.
In \cite{jarvinen2016four}, an implementation of {\fourq} on FPGAs is proposed, which was the first time {\fourq} was implemented and deployed on reconfigurable hardware.
As expected, the speed results of the hardware design of {\fourq} are positive: a speedup factor of 2-2.5 was observed on a Xillinx Zynq-7020 FPGA, in comparison with the corresponding variants of the fastest Curve22519 implementation on the same device.
The proposed {\fourq} hardware implementation exhibits constant time execution, which protects against timing and simple side channel attacks.
This enables us to test the resistance of the proposed hardware implementation against other, more advanced, side channel attacks (see \Cref{chp: Side Channel Attacks}).

\section{Related Work}
Side-channels rely on the relation between information leaked through a side-channel and the secret data related to this information. 
A frequent used side-channel is the power consumption of a device.
A simple attack that makes use of this information is simple power analysis (SPA), in which the power consumption of a device is visually examined.
This enables an adversary to observe the different operations occurring in the execution of the algorithm.
If the algorithm does not run in constant time, the graph of the power consumption can be used to retrieve the secret data in the execution of the algorithm.
This type of attack is however hard to perform in practice, as countermeasures against SPA are generally fairly simple to implement.
A more advanced side-channel attack is differential power analysis (DPA), in which power consumption measurements are statistically analyzed \cite{kocher1999differential}.
This requires a large number of power traces from the same device using the same secret.
The more traces we capture, the higher the chances of successfully performing the attack.
The number of traces required is related to the noise that is inherent to captured power traces.
DPA requires multiple power traces for the same secret, which is something that cannot always be realized in practice.
Therefore, new techniques that fall between SPA and DPA have been developed, with one of the most notable ones being template attacks \cite{chari2002template, rechberger2004practical, choudary2013efficient}.
Template attacks are generally used to attack the secret scalar in a scalar multiplication algorithm.
This attack only requires one target trace to attack this secret, while numerous template traces are needed in the precomputation phase.
This type of attack was improved in \cite{batina2014online}, in which Online Template Attacks (OTAs) were introduced and successfully applied.
OTAs reduce the number of templates and only require one target trace from the device under attack.
These are horizontal SCAs, in which different parts of the same trace are considered to attack different key bits.
The OTA applied in the original paper was used to attack different scalar multiplication algorithms that executed in constant-time.
In addition, different input representations (i.e. affine and projective) for these algorithms were also considered.
They only attacked the first 5 bits of the scalar due to a problem in their measurement setup.
However, a complete key retrieval using an OTA was performed in \cite{dugardin2016dismantling}.
In this paper, the open-source cryptographic library PolarSSL, that can be used in embedded devices, was attacked.
The implementation was modified to speed up the finite field computations.
Simply put, they made use of leakage due to a potential overflow in the field multiplication.
To increase the success rate of the template matching phase, they averaged multiple traces for a single template.
This increased the correlation value of the correct template from 69\% (when only using 1 template trace) to 99.8\% (when using the average of 100 template traces). 
The correlation value was calculated using the Pearson correlation coefficient.
However, the chance of retrieving the 256-bit scalar when using 100 additional template traces per trace still remains low.
Due to their approach, they were however able to correct and detect errors in attacking a single key-bit with reasonable probability.
This made full scalar retrieval very likely.

Most of the published power analysis attacks are applied to smart card and microcontrollers.
\cite{ors2003power} is one of the first papers in which a simple power-analysis attack was applied to FPGAs.
In this paper, a Montgomery modular multiplier implementation (without the final subtraction) on a FPGA was attacked.
The attack involved a visual inspection of the computation power trace, which clearly showed the secret key.
In \cite{standaert2004power}, a DPA attack was applied to a FPGA running a DES implementation. 
As mentioned in this paper, the physical behavior of FPGAs is different than smart cards.
Therefore, the original proposal of DPA and its improvements were not directly applicable to FPGAs.
This was dealt with by generalizing the power model of the attack to account for this difference in physical behavior.
The proposed techniques were successfully applied to DES, and it was also verified that other block ciphers (including AES Rijndael) were vulnerable to the proposed methods.
In \cite{guntur2014side}, a new FPGA board called SAKURA-G was introduced that contains two Spartan-6 FPGAs.
The SAKURA-G board was evaluated by making use of a correlation power analysis (CPA) on an AES circuit with no countermeasures against SCAs.
Results of this power analysis were compared to the previously introduced standard boards SASEBO-GII and SASEBO-G.
The CPA on each of the boards was conducted five times without changing the conditions and using the same key and plaintext.
It was shown that the SAKURA-G board reduced the number of templates needed to successfully perform a CPA by half compared to the other boards \cite{guntur2014side}. 
In addition, power traces captured using the SAKURA-G board were cleaner compared to the other boards. 
These results make the SAKURA-G board a good option when you want to verify the protection of hardware designs against side-channel attacks.

In \cite{costello2015fourq}, a fast curve named {\fourq} was introduced that provides a very fast way to perform scalar multiplication.
An implementation of {\fourq} on microcontrollers with strong countermeasures against side-channel attacks was presented in \cite{liu2017fourq}. 
The countermeasures in this implementation applied to a variety of side-channel attacks, and should increase the difficulty for an attacker to successfully apply an unprofiled vertical attack.
The effectiveness of the countermeasures was partially verified by carrying out a DPA evaluation using a discovery board containing the ARM Cortex-M4 microcontroller. 


\section{Outline Thesis}
In this section, we describe the outline for the rest of this thesis.
In \Cref{chp: SAKURA-G} we provide information about the board that is used to run the hardware implementation of {\fourq} and to capture the power traces necessary to perform the side-channel attack: the Side-channel AttacK User Reference Architecture
Board (SAKURA-G) \cite{guntur2014side}.
This includes details about how the data is transmitted from and to the board and which state machines and interfaces are used to realize this.
Information about elliptic curves is summarized in \Cref{chp: Elliptic Curves}.
In \Cref{chp: FourQ}, we describe the details of the curve {\fourq}.
Details regarding the hardware implementation of {\fourq} are discussed in \Cref{chp: FourQ On Hardware}.
\Cref{chp: Side Channel Attacks} contains details about (online) template attacks.
The application of an OTA to {\fourq} and the corresponding results are described in \Cref{chp: Attacking FourQ}.
\Cref{chp: Conclusion and Discussion} concludes and discusses this thesis.
