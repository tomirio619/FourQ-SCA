% !TeX spellcheck = en_US
% !TeX root = ../Tom_Sandmann-master_thesis
\begin{abstract}
Electronic devices can perform all sorts of operations which we can analyze.
We can either analyze a single operation executed by the device, or a full execution.
Side-channel attacks (SCAs) exploit the relation between information leaked through a side-channel and the corresponding secret data.
Different types of side-channels can be used: timing information, power consumption and electromagnetic signals are all information sources that can be used in SCAs.
When we consider an implementation on a specific device, we can use these side channels to obtain secret information stored or used in this device. 
This can break secrecy assumptions about confidential information used in these devices.
In Elliptic Curve Cryptography (ECC), we use public and private keys to encrypt and decrypt messages.
If we have an encrypted message, the original message can only be obtained if one knows the corresponding private key.
Construction of these keys depends on a problem that makes it particularly hard to obtain the private key assuming  the public key is accessible to anyone.
With respect to ECC, the difficulty of this problem depends on the intractability of determining $k$ from $Q = kP$ where the points $P$ and $Q$ are known. 
This computation of $kP$ is also known as scalar multiplication.
In \cite{costello2015fourq}, a new curve called {\fourq} is introduced.
Scalar multiplication on {\fourq} is very fast compared to other curves that were considered after its introduction.
This is because {\fourq} can make use of a 4-dimensional Gallant-Lambert-Vanstone (GLV) decomposition, which reduces the total number of operations needed to compute a scalar multiplication.
In this thesis, we attack a hardware implementation of {\fourq} on FPGAs which was introduced in \cite{jarvinen2016four}.
We make use of an Online Template Attack (OTA) \cite{batina2014online}, which greatly reduces the number of templates needed compared to regular template attacks. 
\end{abstract}